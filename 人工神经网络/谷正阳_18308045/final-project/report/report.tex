\documentclass[UTF8, a4paper, 11pt]{article}
\usepackage{diagbox}
\usepackage{subfigure}
\usepackage[UTF8, scheme=plain]{ctex}
\usepackage{fontspec}
\usepackage{float}
\usepackage{amsmath}
\newtheorem{myDef}{Definition}
\usepackage{graphicx}
\usepackage{geometry}
\usepackage{listings}
\usepackage{xcolor}
\usepackage{caption}
\geometry{scale=0.8}
\linespread{1.5}
\usepackage{hyperref}
\usepackage{color}
\usepackage{fontspec}
\usepackage{enumitem}
\usepackage[linesnumbered,boxed]{algorithm2e}    
\usepackage{xeCJK}
\usepackage{indentfirst} 
\usepackage{amssymb}
\graphicspath{{Pics/}} 	% 在于.tex同级的目录下创建名为pic的文件夹,存放图片


\setlength{\parindent}{2em}

\lstset{
    language={python},
    frame=shadowbox,
    breaklines=true,
    numbers=left,
    backgroundcolor=\color[RGB]{245,245,244},
    rulesepcolor=\color{red!20!green!20!blue!20},
    numberstyle={\color[RGB]{0,192,192}\tiny},
    basicstyle=\footnotesize \fontspec{Source Code Pro}
}
\setenumerate[1]{itemsep=0pt,partopsep=0pt,parsep=\parskip,topsep=0pt}
\setitemize[1]{itemsep=0pt,partopsep=0pt,parsep=\parskip,topsep=0pt}
\setdescription{itemsep=0pt,partopsep=0pt,parsep=\parskip,topsep=0pt}


\title{	
\normalfont \normalsize
\textsc{School of Data and Computer Science, Sun Yat-sen University} \\ [25pt] %textsc small capital letters
\rule{\textwidth}{0.5pt} \\[0.4cm] % Thin top horizontal rule
\huge report\\ % The assignment title
\rule{\textwidth}{2pt} \\[0.5cm] % Thick bottom horizontal rule
\author{18308045 谷正阳}
\date{\normalsize\today}
}

\begin{document}
\maketitle
\tableofcontents
\newpage
\section{Alternative architecture}
阅读\texttt{layer\_utils},它已经构建了许多多层的结构,包括\texttt{affine\_relu},\texttt{affine\_bn\_relu},\texttt{conv\_relu},\texttt{conv\_bn\_relu},\texttt{conv\_relu\_pool}。
而且它使用的conv是从\texttt{fast\_layer}中最好的\texttt{conv\_fast}(不同卷积层实现的对比间\texttt{fast\_layer}的注释)。
因而我在建立\texttt{AlternativeArchitecture}时,直接使用了\texttt{layer\_utils}。

卷积神经网络的一般结构为:
\begin{equation*}
	\{INPUT\rightarrow[[CONV\rightarrow RELU]\times N\rightarrow POOL?]\times M\rightarrow[FC\rightarrow RELU]*K\rightarrow FC\},
\end{equation*}
一半是卷积部分,一半是全连接部分。
我分别用\texttt{conv\_dims}和\texttt{affine\_dims}表示两者。
其中卷积部分,每一层可能带\texttt{bn}也可能带\texttt{pool},因此设后两个参数分别为\texttt{bn}和\texttt{pool}的参数字典,每个\texttt{conv\_dim}有如下形式:
\begin{lstlisting}
(F, HH, WW, conv_param, bn_param, pool_param)
\end{lstlisting}
全连接部分,每一层可能带\texttt{bn},因此设最后一个参数为\texttt{bn}的参数字典,每个\texttt{affine\_dim}有如下形式:
\begin{lstlisting}
(D, bn_param)
\end{lstlisting}
\section{调参过程}
以markdown语法写在\texttt{ConvolutionalNetworks.ipynb}中。
%\clearpage
%\bibliography{E:/Papers/LiuLab}
%\bibliographystyle{apalike}
\end{document}
%%% Local Variables:
%%% mode: latex
%%% TeX-master: t
%%% End:
