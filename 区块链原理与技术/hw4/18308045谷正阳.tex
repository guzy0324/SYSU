\documentclass[UTF8,12pt]{article}
\usepackage[colorlinks,linkcolor=black]{hyperref}
\usepackage{longtable}
\usepackage{xcolor}
\usepackage{colortbl}
\usepackage{mathtools}
\usepackage{multirow}
\usepackage{amsmath,bm}
\usepackage{amssymb}
\usepackage{float}
\usepackage{ctex}
\usepackage{appendix}
\usepackage{graphicx}
\usepackage{caption}
\usepackage{subcaption}
\usepackage{listings}
\usepackage[a4paper,left=10mm,right=10mm,top=15mm,bottom=15mm]{geometry}
\graphicspath{{Pic/}} 	% 在于.tex同级的目录下创建名为pic的文件夹,存放图片
\title{SPV}
\date{}
\author{18308045 谷正阳}
\begin{document}
\maketitle
\section{SPV的定义}
\subsection{区块结构}
每个区块是包含区块头和区块体的,所有的交易都存在区块体中,而区块头仅记录用区块体的全部交易计算的Merkel树根。因此验证区块体有没有被修改,只需要验证Merkel树根是否和区块头记录的一致。
而整个区块链就是由区块,使用区块头的哈希指针串接而成的链表。
\subsection{SPV}
SPV是一种简化的支付验证过程。它的验证过程如下:
\begin{enumerate}
    \item 节点从区块链网络上获取并存储最长链的所有区块头至本地;
    \item 计算待验证支付的交易哈希值;
    \item 节点从区块链获取待验证支付对应的默克尔树哈希认证路径;(这里找到了该交易对应的哈希值)
    \item 根据哈希认证路径,计算默克尔树的根哈希值,将计算结果与本地区块头中的默克尔树的根哈希值进行比较,定位到包含待验证支付的区块;(找到这个哈希值属于哪个区块)
    \item 根据该区块头所处的位置,验证该区块的区块头是否已经包含在已知最长链中,确定该支付已经得到的确认数量,如果包含则证明支付真实有效。(证明本交易得到了6次确认)
\end{enumerate}
\section{SPV的作用}
SPV不使用在本地存储全部的区块链,来验证交易的真实存在性,即验证交易已经被6次确认过已经被最长链承认了。用它可以实现安装在移动设备上的钱包软件(SPV钱包,仅提供支付确认而非挖矿)。

另外SPV还可以实现双向锚定技术(可以实现暂时的将数字资产在主链中锁定,同时将等价的数字资产在侧链中释放,同样当等价的数字资产在侧链中被锁定的时候,主链的数字资产也可以被释放)进而实现
侧链技术。具体来说,SPV可以证明一个交易确实已经在区块链中发生过,进而实现主链侧链的交互。
\section{SPV对区块链的利弊}
\subsection{利}
SPV由于最多只需要存储全部的区块头,而不需要存储区块体,可以减小空间开销,因而得以实现部署在移动设备上的SPV钱包。另外它也是两条链交互的过程,可以用来实现侧链技术。
\subsection{弊}
主要是安全性的问题。但是SPV因为没有保存全部区块的节点信息,需要和其他节点配合才能进行验证,所以一旦SPV节点连入了一个虚假的网络中的,存在被恶意攻击的风险。另外spv由于没有全部的交易记录
,不能验证某个交易不存在,因而其易受双花攻击的影响。
\end{document}
