\documentclass[UTF8, a4paper, 11pt]{article}
\usepackage{diagbox}
\usepackage{subfigure}
\usepackage[UTF8, scheme=plain]{ctex}
\usepackage{fontspec}
\usepackage{float}
\usepackage{amsmath}
\newtheorem{myDef}{Definition}
\usepackage{graphicx}
\usepackage{geometry}
\usepackage{listings}
\usepackage{xcolor}
\usepackage{caption}
\geometry{scale=0.8}
\linespread{1.5}
\usepackage{hyperref}
\usepackage{color}
\usepackage{fontspec}
\usepackage{enumitem}
\usepackage[linesnumbered,boxed]{algorithm2e}    
\usepackage{xeCJK}
\usepackage{indentfirst} 
\graphicspath{{Pics/}} 	% 在于.tex同级的目录下创建名为pic的文件夹,存放图片


\setlength{\parindent}{2em}

\lstset{
    language={python},
    frame=shadowbox,
    breaklines=true,
    numbers=left,
    backgroundcolor=\color[RGB]{245,245,244},
    rulesepcolor=\color{red!20!green!20!blue!20},
    numberstyle={\color[RGB]{0,192,192}\tiny},
    basicstyle=\footnotesize \fontspec{Source Code Pro}
}
\setenumerate[1]{itemsep=0pt,partopsep=0pt,parsep=\parskip,topsep=0pt}
\setitemize[1]{itemsep=0pt,partopsep=0pt,parsep=\parskip,topsep=0pt}
\setdescription{itemsep=0pt,partopsep=0pt,parsep=\parskip,topsep=0pt}


\title{	
\normalfont \normalsize
\textsc{School of Data and Computer Science, Sun Yat-sen University} \\ [25pt] %textsc small capital letters
\rule{\textwidth}{0.5pt} \\[0.4cm] % Thin top horizontal rule
\huge 数电实验9\\ % The assignment title
\rule{\textwidth}{2pt} \\[0.5cm] % Thick bottom horizontal rule
\author{18308045 谷正阳}
\date{\normalsize\today}
}

\begin{document}
\maketitle
\tableofcontents
\newpage
\section{仿真实验}
\subsection{汽车尾灯}
\subsubsection{电路图}
\begin{figure}[H]
    \centering
    \includegraphics[width=0.8\textwidth]{ex10.1电路图.png}
\end{figure}
本质是一个Johnson counter,从左到右标号分别为$0,1,2,3$,右移状态为$0111$时清零,左移状态为$1110$时清零。
假设$\overline{CLR}$用于控制清零,$\overline{RIGHT}/LEFT$用于控制左移右移,可得:
$$\overline{R_i}=\overline{(Q_0\oplus Q_3)\cdot(Q_3\oplus LEFT)}\cdot\overline{CLR}$$
。另外J-K触发器的$J_i,\overline{K_i}$按照Johnson counter连,特别地,要用二选一选择器选择从左还是右获得数据。
\subsubsection{演示视频}
\href{1.mp4}{点击观看演示视频1.mp4}
\subsection{小键盘}
\subsubsection{电路图}
\begin{figure}[H]
    \centering
    \includegraphics[width=0.8\textwidth]{ex10.2电路图.png}
\end{figure}
由于仿真运算量过大,仿真时间流逝比真实时间慢很多,可能按键时间太短但第二个单稳态触发器脉冲未结束。
因而将单稳态触发器的单稳态时间常数调到$1\mu$。
\subsubsection{演示视频}
\href{2.mp4}{点击观看演示视频2.mp4}
\section{实验箱实验}
\subsection{汽车尾灯}
\href{3.MP4}{点击观看演示视频3.MP4}
%\clearpage
%\bibliography{E:/Papers/LiuLab}
%\bibliographystyle{apalike}
\end{document}
%%% Local Variables:
%%% mode: latex
%%% TeX-master: t
%%% End:
